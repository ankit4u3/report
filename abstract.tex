\begin{document}
\begin{center}
\thispagestyle{empty}
\vspace*{4\baselineskip}
\LARGE{\textbf{ABSTRACT}}\\[1.0cm]
\end{center}
\thispagestyle{empty}
\large{\emph{Thoughtcloud.It  is a social thinking platform that intends to bring people together by the way they think. The concept came into picture by a thought about connecting people who are thinking about the exact same thing at the same time.ThoughtCloud.It relies its core strength on -Preciseness - Restricted to three words long thoughts only.Lucidity - Easy and interactive design makes it a child's play to use it. Innovation - Innovative feature of real time thought mapping by demography. Thought Cloud c:geo is a simple to use but powerful geocaching client with a lot of additional features. All you need to get started is an account on geocaching.com. Find caches using the live map or by using one of the many search functions. Navigate to a cache or a waypoint of a cache with the built-in compass function, the map or hand over the coordinates to various external apps (e.g. Radar, Google Navigation, StreetView, Locus, Navigon, Sygic and many more).
Store cache information to your device directly from geocaching.com as well as via GPX file import to have it available whenever you want. You can manage your stored caches in different lists and can sort and filter them according to your needs Stored caches together with offline map files or static maps can be used to find caches without an internet connection (e.g. when roaming).
Logs can be posted online or stored offline for later submission or exported via field notes.Search and discover trackables, manage your trackable inventory and drop a trackable while posting a cache log.\\[1cm]}}

